
\myChapter{Introduzione}
%\section{Panoramica}
%Negli ultimi anni, grazie all'avvento e alla ribalta di nuove discipline e orizzonti per l'informatica come il machine learning e l'intelligenza artificiale, stanno ritornando in auge gli algoritmi genetici. La loro nascita si attesta negli anni '60 da parte di un team di ricerca il cui erede spirituale, oltre che pietra miliare in tale campo, sta nella figura di Goldman.\breakGli algoritmi genetici (che da ora in poi chiameremo GA per semplicità).
Il sempre maggiore interesse verso il machine learning, congiuntamente alla crescita esponenziale dell'importanza dell'intelligenza artificiale,  hanno riportato alla luce gli algoritmi genetici introdotti per la prima volta da J. Holland nel 1960 ed ampliamente analizzati dal successore di costui, D. E. Goldeberg \cite{goldberg1} nel 1989. Tali algoritmi stanno ritornando in auge grazie alla loro semplice implementazione ed alla loro adattablit\`a, il tutto combinato alla loro robustezza nel trovare la soluzione ottimale ad un problema di ricerca e/o ottimizzazione.

%\section{Seconda sezione}
%Testo.
Gli algoritmi genetici sono basati sui meccanismi della selezione naturale implementando scambi di informazioni casuali e deterministici combinati alla sopravvivenza degli "individui" pi\`u idonei; seppur l'idea su cui essi hanno le fondamenta e la loro implementazione siano relativamente pi\`u semplici rispetto ai metodi tradizionali, lo studio degli algoritmi genetici non \`e assolutamente meno degno di nota.
\vspace{3mm}

Lo scopo che si prefigge questa tesi \`e dare una visione precisa del funzionamento degli algoritmi genetici e delle loro caratteristiche peculiari, il tutto supportato tramite esempi pratici e, al tempo stesso, metterne in luce i vantaggi e gli svantaggi che ne derivano; da qui in avanti ci riferiremo agli algoritmi genetici con l'acronimo, proveniente dall'inglese, GA.

La tesi, dopo questa prima introduzione, \`e strutturata nel seguente modo:
\begin{enumerate}
%\setcounter{enumerate}{2}
    \item Il capitolo 2, dopo una breve introduzione storica, tratta nel dettaglio i principi fondamentali dei GA ed il loro funzionamento, illustrando le funzioni che li caratterizzano come tali.
    \item Il capitolo 3 porta un esempio semplice di applicazione di un GA, oltre ad evidenziare svantaggi e vantaggi di questo approccio.
    %\item Il capitolo 4 illustra il funzionamento di un GA con il compito di ordinare un array confrontando le sue prestazioni con i pi\`u famosi algoritmi di ordinamento.
    \item Il capitolo 4 illustra il problema del cammino minimo in un grafo con annesso riferimento all'algoritmo di Dijkstra, in seguito mostra il lavoro eseguito dal GA costruito per lo stesso compito; per ultimo, i due algoritmi vengono messi a confronto con maggiore approfondimento dei pro e contro del GA in tale problema ed affini.
    \item Il capitolo 5 discute su quanto visto nei capitoli precedenti, arrivando a conclusioni volte a migliorare gli esempi di implementazioni illustrati. Al tempo stesso, daremo una visione pi\`u ampia sui campi di applicazione dei GA, mostrando come essi stiamo attirando un sempre maggiore interesse, concludendo il tutto con una visione sul futuro di tale approccio.
    \item Dopo i necessari ringraziamenti, al termine della tesi (in un'apposita appendice) si potranno osservare i codici usati nell'arco di tutti i capitoli della tesi.
    %\item Il capitolo 5 tratta dell'uso dei GA con i grafi (colorazione, ricerca del cammino minimo, etc...), con una parentesi riguardo il problema del commesso viaggiatore.
    %\item Il capitolo 6, in conclusione, mostra la risoluzione di sudoku attraverso i GA e l'importanza di tutto questo.
\end{enumerate}
\newpage