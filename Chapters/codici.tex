
\appendix
\myChapter{Codici}
\section{Capitolo 3}
\subsection{Launcher}
\begin{lstlisting}[style=matlab, style=matlab2, style=matlab3]
x=[0.00:0.01:10.23]';
n=length(x);
y=zeros(n, 1);
y(1:n)=(x(1:n)-7).^2+1; %funzione da minimizzare

popSize=30; %dimensione della popolazione
pCrossover=0.75; %coefficiente di crossover
pMutation=0.0333; %coefficiente di mutazione
maxGen=500; %generazione massima

%popolazione iniziale generata casualmente
p=fix(x(n).*rand(popSize,1)*100)/100;

%inizializzazione e calcolo fitness inziale
fitness=zeros(popSize, 1);
fitness(1:popSize)=(p(1:popSize)-7).^2+1;

%# di generazione in cui il fitness non cambia
cFit=0;

[minFit, minIndex]=min(fitness);
minPoint=p(minIndex);
for i=1:maxGen
    %stampa a video
    fprintf("Generazione: %d, Individuo migliore: %d, 
        Fitness minima: %d, Fitness media: %d\n", i, 
        minPoint, minFit, mean(fitness));
    
    [z,fitness]=GAMatlab(pCrossover,pMutation,p,
        fitness);
    if min(fitness)>=minFit
        cFit=cFit+1;
    else
        [minFit, minIndex]=min(fitness);
        minPoint=z(minIndex);
        cFit=0;
    end
    population=z;
    if cFit==100
        fprintf("Limite massimo raggiunto");
        %potremmo usare un metodo tradizionale
        break; 
    end
end
\end{lstlisting}
\subsection{Function principale ed operatori}
\begin{lstlisting}[style=matlab, style=matlab2, style=matlab3]
function [newPopulation,fitness]=GAMatlab(pCrossover,
pMutation,p,fitness)
%
% [newPopulation,fitness]=GAMatlab(pCrossover,
% pMutation,population,fitness)
% Genera la nuova popolazione attraverso gli
% operatori fondamentali dei GA
n=length(population);
chromosomes=zeros(n,10); 
newChromosomes=zeros(n,10);
newPopulation=zeros(n,1);
k=1; %numero individui della nuova generazione
[fitMin,minIndex]=min(fitness); %fitness minimo
for i=1:n
    %codifica
    chromosomes(i,1:10)=encode(p(i));
end
while k<n+1
    %selezione
    [parent1,chromosomes2,j]=selection(chromosomes,
        fitness);
    fitness2=fitness([1:j-1 j+1:end]);
    parent2=selection(chromosomes2,fitness2);
   
    %crossover
    if (rand(1)<=pCrossover)
        [son1,son2]=crossover(parent1,parent2);
        
        %mutazione
        son1=mutation(pMutation, son1);
        son2=mutation(pMutation, son2);
        
        %aggiunta all'insieme dei nuovi cromosomi
        newChromosomes(k,1:10)=son1;
        newChromosomes(k+1,1:10)=son2;
        k=k+2;
    end
end
for i=1:n
    %decodifica
    newPopulation(i)=decode(newChromosomes(i,1:10));
end
%calcolo del fitness della nuova popolazione
fitness(1:n)=(newPopulation(1:n)-7).^2+1;

%inserisco il miglior individuo della precedente
%generazione al posto del peggiore di quella attuale
[fitMax, maxIndex]=max(fitness);
newPopulation(maxIndex)=p(minIndex);
fitness(maxIndex)=fitMin;
end

%
%Selezione a roulette
%
function [parent,chromosomes2,j]=
    selection(chromosomes,fitness)
sumFit=sum(fitness(1:length(fitness)));
n=length(chromosomes);
partSumFit=0; %somma delle parti
i=1;
randomPoint=rand(1)*sumFit; %numero casuale
while partSumFit<randomPoint && i<n
    partSumFit=partSumFit+fitness(i);
    i=i+1;
end
parent=zeros(1, 10);
parent=chromosomes(i, 1:10);
if nargout>1
    chromosomes2=chromosomes([1:i-1 i+1:end], 1:10);
    j=i;
end
end

%
%Crossover a punto singolo
%
function [son1,son2]=crossover(parent1,parent2)
son1=zeros(1,10);
son2=zeros(1,10);
crossPoint=randi(9, 1);
son1(1:crossPoint)=parent1(1:crossPoint);
son1(crossPoint+1:end)=parent2(crossPoint+1:end);
son2(1:crossPoint)=parent2(1:crossPoint);
son2(crossPoint+1:end)=parent1(crossPoint+1:end);
end

%
%Mutazione
%
function mutChromosome=mutation(pMutation,chromosome)
for i=1:length(chromosome)
    if rand(1)<=pMutation
        chromosome(i)=1-chromosome(i);
    end
end
mutChromosome=chromosome;
end

%
%Codifica
%
function chromosome=encode(individual)
s=fix(individual*100);
output=zeros(1,10);
i=10;
while s>0
    output(1,i)=mod(s, 2);
    s=fix(s/2);
    i=i-1;
end
chromosome=output;
end

%
%Decodifica
%
function individual=decode(chromosome)
individual=0;
for i=10:-1:1
    individual=individual+chromosome(i)*(2^(10-i));
end
individual=individual*0.01;
end
\end{lstlisting}
\section{Capitolo 4}
\subsection{Classe Node}
\begin{lstlisting}[style=Java]
package geneticalgorithm;

import java.util.LinkedList;
import java.util.HashMap;
import java.util.Iterator;

public class Node {
	private String name;
	private LinkedList<Node> path; //cammino minimo dal nodo iniziale
	private int d; //distanza dal nodo sorgente
	private HashMap<Node, Integer> mapNodes; //nodi adiacenti e loro distanza dal nodo stesso
	private int id; //id del nodo, necessario per la matrice di adiacenza
	private static int lastId=0; //ultimo id nodo salvato
	
	public Node(String name) {
		this.name=name;
		path=new LinkedList<>();
		d=Integer.MAX_VALUE;
		mapNodes=new HashMap<>();
		id=lastId++;
	}
	
	/**
	 * Connette il nodo ad un altro nodo n impostando al tempo stesso il peso dell'arco che li congiunge
	 * @param n
	 * @param distance
	 * @param alreadyConnected
	 */
	public void connectTo(Node n, int distance, boolean alreadyConnected) {
		mapNodes.put(n, distance);
		if (!alreadyConnected) n.connectTo(this, distance, true);
	}
	
	/**
	 * Restituisce tutti i nodi adiacenti a tale nodo
	 * @return una HashMap<Node, Integer> di nodi
	 */
	public HashMap<Node, Integer> getConnectedNodes() {
		return mapNodes;
	}
	
	/**
	 * Restituisce il tragitto per arrivare al nodo, usato nell'algoritmo di Dijkstra
	 * @return un iterator di nodi
	 */
	public Iterator<Node> getPath() {
		return path.iterator();
	}
	
	/**
	 * Aggiunge un nodo al cammino per arrivare al nodo in question
	 * @param itPath
	 * @param n
	 */
	public void addToPath(Iterator<Node> itPath, Node n) {
		path.removeAll(path);
		while (itPath.hasNext()) {
			Node m=itPath.next();
			path.add(m);
		}
		path.add(n);
		//path.add(this);
	}
	
	/**
	 * Imposta la distanza cumulativa dal nodo di partenza
	 * @param d
	 */
	public void setDistance(int d) {
		this.d=d;
	}
	
	public int getDistance() {
		return d;
	}
	
	public String getName() {
		return name;
	}
	
	public int getId() {
		return id;
	}
}
\end{lstlisting}
\subsection{Classe Chromosome}
\begin{lstlisting}[style=Java]
package geneticalgorithm;

import java.util.Iterator;
import java.util.ArrayList;
import java.util.Comparator;

public class Chromosome {
	private ArrayList<Node> genes; //lista dei geni
	private int fitness=Integer.MAX_VALUE; //valore di fitness
	
	public Chromosome() {
		genes=new ArrayList<>();
	}
	
	/**
	 * Imposta i geni del cromosoma a partire da un iteratore
	 * @param genes
	 */
	public Chromosome(Iterator<Node> genes) {
		this.genes=new ArrayList<>();
		while (genes.hasNext()) {
			this.genes.add(genes.next());
		}
	}
	
	/**
	 * Restituisce i geni del cromosoma
	 * @return un iterator di geni, ovvero i nodi del grafo
	 */
	public Iterator<Node> getGenes() {
		return genes.iterator();
	}
	
	public int getFitness() {
		return fitness;
	}
	
	public void setFitness(int fitness) {
		this.fitness=fitness;
	}
	
	public int getSize() {
		return genes.size();
	}
	
	/**
	 * Scorrendo tutta la lista dei geni, controlla se il cromosoma corrisponde ad un altro creato in precedenza,
	 * metodo utilizzato solamente nella costruzione della prima generazione
	 * @param c
	 * @return un valore booleano indicante se il cromosoma ha gli stessi geni di un altro cromosoma
	 */
	public boolean hasSamePath(Chromosome c) {
		boolean b=true;
		if (genes.size()==c.getSize()) {
			Iterator<Node> itGenes1=c.getGenes();
			Iterator<Node> itGenes2=getGenes();
			while (itGenes1.hasNext()) {
				Node n1=itGenes1.next();
				Node n2=itGenes2.next();
				if (!n1.equals(n2)) b=false;
			}
			//for(Node)
		} else {
			b=false;
		}
		return b;
	}
	
	@Override
	public String toString() {
		String s="";
		for (Node n: genes) {
			s+=n.getName()+" ";
		}
		return s;
	}
	
	public static Comparator<Chromosome> fitnessComparator = new Comparator<Chromosome>() {
		@Override
		public int compare(Chromosome c1, Chromosome c2) {
			return (c1.getFitness()<c2.getFitness() ? -1 : (c1.getFitness()==c2.getFitness() ? 0 : 1));
		}
	};
 }
\end{lstlisting}
\subsection{Classe Graph}
\begin{lstlisting}[style=Java]
package geneticalgorithm;

import java.util.ArrayList;
import java.util.Iterator;
import java.util.HashMap;
import java.util.Map.Entry;
import java.util.Random;

public class Graph {
	private ArrayList<Node> nodes; //nodi del grafo
	private Population population;
	
	public Graph() {
		nodes=new ArrayList<>();
	}
	
	/**
	 * Aggiunge nodi al grafo
	 * @param n
	 * @throws Exception
	 */
	public void addNode(Node... n) throws Exception{
		for (Node m: n) {
			for (Node k: nodes) {
				if (k.equals(m)) throw new Exception("Il nodo appartiene gi\u00E0 al grafo");
			}
			nodes.add(m);
		}
	}
	
	/**
	 * Partendo da una lista di nodi, restituisce quello con valore di distanza minimo
	 */
	private Node minDistanceNode(ArrayList<Node> nvNodes) {
		Node minNode=null;
		int minDistance=Integer.MAX_VALUE;
		for (Node n: nvNodes) {
			int d=n.getDistance();
			if (d<minDistance) {
				minDistance=d;
				minNode=n;
			}
		}
		return minNode;
	}
	
	/**
	 * Imposta la distanza del nodo destinazione ad un nuovo valore se e solo se
	 * quello precedente risulta maggiore della somma fra l'intero dato passato
	 * come argomento e la distanza cumulativa del nodo destinazione stesso
	 * @param destination
	 * @param source
	 * @param distance
	 */
	private void setDistance(Node destination, Node source, int distance) {
		int sDistance=source.getDistance();
		if (sDistance+distance< destination.getDistance()) {
			destination.setDistance(sDistance+ distance);
			Iterator<Node> path=source.getPath();
			destination.addToPath(path, source);
		}
	}
	
	/**
	 * Calcola il cammino minimo utilizzando l'algoritmo di Dijkstra a partire da un nodo
	 * inziiale verso uno finale
	 * @param source
	 * @param destination
	 * @return il cammino minimo da sorgente a destinazione
	 * @throws Exception
	 */
	public Iterator<Node> DijkstraAlgorithm(Node source, Node destination) throws Exception{
		if (!nodes.contains(source) || !nodes.contains(destination)) throw new Exception("Il nodo definito come iniziale o finale non appartiene al grafo");
		
		//Passo 1 e 2
		source.setDistance(0);
		ArrayList<Node> vNodes=new ArrayList<>();
		ArrayList<Node> nvNodes=new ArrayList<>();
		nvNodes.add(source);
		
		while (nvNodes.size()!=0) {
			Node n=minDistanceNode(nvNodes);
			nvNodes.remove(n);
			if (n.equals(destination)) break; //se siamo giunti a destinazione terminiamo, Passo 5
			
			HashMap<Node, Integer> map=n.getConnectedNodes();
			
			//Passo 3
			for (Entry<Node, Integer> e: map.entrySet()) {
				Node m=e.getKey();
				if (vNodes.contains(m)) continue; //necessaria per grafi non orientati
				int d=e.getValue().intValue();
				//if (!nvNodes.contains(m)) {
					setDistance(m, n, d);
					nvNodes.add(m);
				//}
			}
			
			//Passo 4
			vNodes.add(n);
		}
		return destination.getPath();
	}
	
	/**
	 * Crea un cammino casuale da sorgente a destinazione
	 * @param source
	 * @param destination
	 * @return un iteratore di nodi contenente il cammino casuale
	 */
	public Iterator<Node> createRandomPath(Node source, Node destination) {
		boolean dest=false;
		//ArrayList<Node> nvNodes=nodes.forEach();
		ArrayList<Node> path=new ArrayList<>();
		path.add(source);
		ArrayList<Node> vNodes=new ArrayList<>();
		vNodes.add(source);
		while (dest==false) {
			Node n=path.get(path.size()-1);
			HashMap<Node, Integer> map=n.getConnectedNodes();
			ArrayList<Node> connNodes=new ArrayList<>();
			map.entrySet().forEach(e -> connNodes.add(e.getKey()));
			connNodes.removeIf(k -> vNodes.contains(k));
			if (connNodes.isEmpty()) {
				path.remove(n);
				continue;
			}
			Random r=new Random();
			int randomIndex= r.nextInt(connNodes.size());
			Node m=connNodes.get(randomIndex);
			if (m.equals(destination)) dest=true;
			path.add(m);
			vNodes.add(m);
		}
		return path.iterator();
	}
	
	/**
	 * Calcola il cammino minimo da sorgente a destinazione usando il GA
	 * @param source
	 * @param destination
	 * @param pSize
	 * @param maxGen
	 * @param pCrossover
	 * @param pMutation
	 * @return il cromosoma migliore dell'ultima generazione
	 * @throws Exception
	 */
	public Chromosome ShortestPathGA(Node source, Node destination, int pSize, int maxGen, double pCrossover, double pMutation) throws Exception{
		if (nodes.size()==0) throw new Exception("Non sono stati inseriti nodi nel grafo");
		
		//Creo la popolazione e costruisco la matrice di adiacenza
		population=new Population(nodes.size(), pSize, this);
		population.buildMatrixDist(nodes.iterator());
		
		int j=0;
		while (j<pSize) {
			//Creo un cromosoma assegnandoli un percorso (geni) casuale fra sorgente e destinazione
			Chromosome c=new Chromosome(createRandomPath (source, destination));
			Iterator<Chromosome> itC=population.getChromosomes();
			
			//Controllo se tale percorso esiste già
			boolean hasSamePath=false;
			while (itC.hasNext()) {
				if (itC.next(). hasSamePath(c)) {
					hasSamePath=true;
					break;
				}
			}
			if (!hasSamePath) {
				//Nel caso il cromosoma fosse unico (percorso differente dagli altri), lo aggiungo alla popolazione
				population.addChromosome(c);
				j++;
			}
		}
		
		//Eseguo il GA per un numero massimo di generazioni
		for (int i=0;i<maxGen; i++) {
			population.pathGA(pCrossover, pMutation);
		}
		return population.getFittest(); //Restituzione del miglior cromosoma
	}
}
\end{lstlisting}
\subsection{Classe Population}
\begin{lstlisting}[style=Java]
package geneticalgorithm;

import java.util.ArrayList;
import java.util.Collections;
import java.util.Iterator;
import java.util.HashMap;
import java.util.Map.Entry;
import java.util.Random;
import java.util.stream.Collectors;

public class Population {
	private ArrayList<Chromosome> chromosomes;
	private int bestFitness=Integer.MAX_VALUE; //miglior valore di fitness
	private Chromosome best=null; //miglior individuo
	private int[][] matDist; //matrice delle distanze
	private int pSize; //dimensione popolazione
	private Graph graph;
	private int nCrossover=0; //numero di crossover
	private int nMutation=0; //numero di mutazioni
	
	public Population(int size, int pSize, Graph graph) {
		chromosomes=new ArrayList<>();
		matDist=new int[size][size];
		for (int i=0;i<matDist.length;i++) {
			for (int j=0;j<matDist.length;j++) {
				matDist[i][j]=100;
			}
		}
		this.pSize=pSize;
		this.graph=graph;
	}
	
	/**
	 * Aggiunge un cromosoma alla popolazione
	 * @param c
	 */
	public void addChromosome(Chromosome c) {
		chromosomes.add(c);
		setFitness(c);
	}
	
	/**
	 * Imposta il valore di fitness di un cromosoma
	 * @param c
	 */
	private void setFitness(Chromosome c) {
		int fitness=0;
		Iterator<Node> itNodes=c.getGenes();
		int idPrev=itNodes.next().getId();
		while (itNodes.hasNext()) {
			int idNext=itNodes.next().getId();
			fitness+=matDist[idPrev][idNext];
			idPrev=idNext;
		}
		c.setFitness(fitness);
	}
	
	/**
	 * Restituisce il cromosoma con il miglior valore di fitness
	 * @return il cromosoma migliore
	 */
	public Chromosome getFittest() {
		return best;
	}
	
	/**
	 * Costruisce la matrice di adiacenza
	 * @param itNodes
	 */
	public void buildMatrixDist(Iterator<Node> itNodes) {
		while (itNodes.hasNext()) {
			Node n=itNodes.next();
			HashMap<Node, Integer> map=n.getConnectedNodes();
			int idN=n.getId();
			for (Entry<Node, Integer> e: map.entrySet()) {
				Node m=e.getKey();
				int dist= e.getValue().intValue();
				int idM=m.getId();
				if (matDist[idN][idM]==100) {
					matDist[idN][idM]= dist;
					matDist[idM][idN]= dist;
				}
			}
		}
	}
	
	/**
	 * Operatore di selezione, applica la sua variante a torneo
	 * @return due cromosomi genitori
	 */
	private ArrayList<Chromosome> selection() {
		//Dimensione del torneo
		int tSize=5;
		
		//Lista temporanea dei cromosomi
		ArrayList<Chromosome> tChroms=new ArrayList<>();
		
		chromosomes.forEach(c -> tChroms.add(c));
		Random r=new Random();
		
		//Cromosomi genitori
		ArrayList<Chromosome> parents=new ArrayList<>();
		
		while (parents.size()<2) {
			//Cromosomi partecipanti al torneo
			ArrayList<Chromosome> selChroms=new ArrayList<>();
			while (selChroms.size()<tSize) {
				int rInt= r.nextInt(tChroms.size());
				Chromosome c=tChroms.get(rInt);
				if (!selChroms.contains(c)) {
					selChroms.add(c);
				}
			}
			
			Collections.sort(selChroms, Chromosome.fitnessComparator);
			
			//Scelta del vincitore del torneo
			double p=0.75;
			double[] partialSums=new double[tSize];
			double sumP=0;
			for (int i=0;i<tSize;i++) {
				sumP+=p*Math.pow((1-p), i);
				partialSums[i]=sumP;
			}
			double rDouble=r.nextDouble()*sumP;
			int i=0;
			while (partialSums[i]<rDouble) {
				i++;
			}
			parents.add(selChroms.get(i));
			tChroms.remove(i);
		}
		return parents;
	}
	
	/**
	 * Operatore di crossover, nella sua variante a punto singolo
	 * @param parents
	 * @return una lista contenente i geni dei figli
	 */
	private ArrayList<ArrayList<Node>> crossover(ArrayList<Chromosome> parents) {
		//Estrazione dei due genitori dalla lista
		Chromosome parent1=parents.get(0);
		Chromosome parent2=parents.get(1);
		
		ArrayList<Node> son1=new ArrayList<>();
		ArrayList<Node> son2=new ArrayList<>();
		
		//Ottenimento dei geni dei due genitori
		Iterator<Node> itP1=parent1.getGenes();
		Iterator<Node> itP2=parent2.getGenes();
		
		//Calcolo della dimensione minima
		int minSize=parent1.getSize();
		if (parent2.getSize()<minSize) minSize=parent2.getSize();
		
		//Generazione del punto di crossover
		Random r=new Random();
		int rIndex=r.nextInt(minSize-1);
		
		//Crossover a punto singolo
		int i=0;
		while (i<rIndex) {
			son1.add(itP1.next());
			son2.add(itP2.next());
			i++;
		}
		while (itP1.hasNext()) {
			son2.add(itP1.next());
		}
		while (itP2.hasNext()) {
			son1.add(itP2.next());
		}
		
		//Restituzione dei due figli
		ArrayList<ArrayList<Node>> sons=new ArrayList<>();
		sons.add(son1);
		sons.add(son2);
		return sons;
	}
	
	/**
	 * Operatore di mutazione
	 * @param individuals
	 * @param pMutation
	 */
	private void mutation(ArrayList<ArrayList<Node>> individuals, double pMutation) {
		Random r=new Random();
		for (int i=0;i<individuals.size();i++) {
			ArrayList<Node> son=individuals.get(i);
			for (int j=0;j<son.size()-1;j++) {
				double rDouble=r.nextDouble();
				
				//Se la mutazione avviene, trovo un nuovo percorso dal nodo scelto verso la destinazione
				if (rDouble<pMutation) {
					nMutation++;
					Node n=son.get(j);
					Iterator<Node> itN= graph. createRandomPath(n, son.get (son.size()-1));
					
					//Eliminimo gli elementi successivi al nodo che ha subito la mutazione
					for (int k= son.size()-1;k>= j;k--) {
						son.remove(k);
					}
					
					//Aggiungo il percorso
					while (itN.hasNext()) son.add(itN.next());
					break;
				}
			}
			
			//Elimino i nodi dupilcati
			son=(ArrayList<Node>)son.stream() .distinct().collect(Collectors. toList());
		}
	}
	
	/**
	 * Creazione della nuova generazione attraverso i tre operatori fondamentali
	 * @param pCrossover
	 * @param pMutation
	 */
	public void pathGA(double pCrossover, double pMutation) {
		ArrayList<Chromosome> newGeneration=new ArrayList<>();
		while (newGeneration.size()<pSize) {
			//Selezione
			ArrayList<Chromosome> parents=selection();
			
			//Generazione di un numero casuale fra 0 ed 1
			Random r=new Random();
			double rDouble=r.nextDouble();
			
			//Crossover
			if (rDouble<pCrossover) {
				nCrossover++;
				ArrayList<ArrayList<Node>> sons=crossover(parents);
				
				//Mutazione
				mutation(sons, pMutation);
				
				//Aggiunta dei nuovi cromosomi alla popolazione
				newGeneration.add(new Chromosome(sons. get(0).iterator()));
				newGeneration.add(new Chromosome(sons. get(1).iterator()));
				
			}
		}
		newGeneration.forEach(c -> setFitness(c));
		Collections.sort(newGeneration, Chromosome.fitnessComparator);
		int i=newGeneration.size()-1;
		while (newGeneration.size()>pSize) {
			newGeneration.remove(i);
			i--;
		}
		chromosomes=newGeneration;
		best=chromosomes.get(0);
		bestFitness=best.getFitness();
	}
	
	@Override
	public String toString() {
		String s="";
		for (Chromosome c: chromosomes) {
			s+=c.toString()+" fitness: "+c.getFitness()+"\n";
		}
		return s;
	}
	
	public int getBestFitness() {
		return bestFitness;
	}
	
	public String getStats() {
		String s="Numero di crossover: "+nCrossover+", numero di mutazioni: "+nMutation;
		return s;
	}
	
	public Iterator<Chromosome> getChromosomes() {
		return chromosomes.iterator();
	}
}
\end{lstlisting}
%\section{Capitolo 5}
\newpage